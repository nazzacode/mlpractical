%% Template for MLP Coursework 2 / 22 November 2021 

%% Based on  LaTeX template for ICML 2017 - example_paper.tex at 
%%  https://2017.icml.cc/Conferences/2017/StyleAuthorInstructions

\documentclass{article}
\usepackage[T1]{fontenc}
\usepackage{amssymb,amsmath}
\usepackage{txfonts}
\usepackage{microtype}

% For figures
\usepackage{graphicx}
\usepackage{subcaption} 

% For citations
\usepackage{natbib}

% For algorithms
\usepackage{algorithm}
\usepackage{algorithmic}

% For internal refercnes
\usepackage{nameref}

% the hyperref package is used to produce hyperlinks in the
% resulting PDF.  If this breaks your system, please commend out the
% following usepackage line and replace \usepackage{mlp2017} with
% \usepackage[nohyperref]{mlp2017} below.
\usepackage{hyperref}
\usepackage{url}
\urlstyle{same}

\usepackage{color}
\usepackage{booktabs} % To thicken table lines
\usepackage{multirow} % Multirow cells in table

% Packages hyperref and algorithmic misbehave sometimes.  We can fix
% this with the following command.
\newcommand{\theHalgorithm}{\arabic{algorithm}}


% Set up MLP coursework style (based on ICML style)
\usepackage{mlp2021}
\mlptitlerunning{MLP Coursework 1 (\studentNumber)}
\bibliographystyle{icml2017}
\usepackage{bm,bbm}


\DeclareMathOperator{\softmax}{softmax}
\DeclareMathOperator{\sigmoid}{sigmoid}
\DeclareMathOperator{\sgn}{sgn}
\DeclareMathOperator{\relu}{relu}
\DeclareMathOperator{\lrelu}{lrelu}
\DeclareMathOperator{\elu}{elu}
\DeclareMathOperator{\selu}{selu}
\DeclareMathOperator{\maxout}{maxout}
\newcommand{\bx}{\bm{x}}



% \definecolor{red}{rgb}{0.95,0.4,0.4}
\definecolor{red}{rgb}{0.4,0.05,0.05}
\definecolor{blue}{rgb}{0.4,0.4,0.95}
\definecolor{orange}{rgb}{1, 0.65, 0}

% \newcommand{\youranswer}[1]{{\color{red} \bf[#1]}} %your answer:
\newcommand{\youranswer}[1]{{\color{red} [#1]}} %your answer:


%% START of YOUR ANSWERS
%% REPLACE sXXXXXXX with your student number
\def\studentNumber{s1869292}



%% START of YOUR ANSWERS
%% Add answers to the questions below, by replacing the text inside the brackets {} for \youranswer{ "Text to be replaced with your answer." }. 
%
% Do not delete the commands for adding figures and tables. Instead fill in the missing values with your experiment results, and replace the images with your own respective figures.
%
% You can generally delete the placeholder text, such as for example the text "Question Figure 3 - Replace the images ..." 
%
% There are 5 TEXT QUESTIONS. Replace the text inside the brackets of the command \youranswer with your answer to the question.
%
% There are also 3 "questions" to replace some placeholder FIGURES with your own, and 1 "question" asking you to fill in the missing entries in the TABLE provided. 
%
% NOTE! that questions are ordered by the order of appearance of their answers in the text, and not necessarily by the order you should tackle them. You should attempt to fill in the TABLE and FIGURES before discussing the results presented there. 
%
% NOTE! If for some reason you do not manage to produce results for some FIGURES and the TABLE, then you can get partial marks by discussing your expectations of the results in the relevant TEXT QUESTIONS. The TABLE specifically has enough information in it already for you to draw meaningful conclusions.
%
% Please refer to the coursework specification for more details.


%% - - - - - - - - - - - - TEXT QUESTIONS - - - - - - - - - - - - 

%% Question 1:
\newcommand{\questionOne} {
\youranswer{Question 1 - Use Figures 1, 2, and 3 to identify the Vanishing Gradient Problem (which of these model suffers from it, and what are the consequences depicted?).

The average length for an answer to this question is approximately 1/5 of the columns in a 2-column page}
}

%% Question 2:
\newcommand{\questionTwo} {
\youranswer{Question 2 - Describe Batch Normalization (BN) by using equations. Consider both training and test time and explain how BN addresses the vanishing gradient problem. Note that you are not required to provide the derivation of gradients w.r.t. weights for BN weights.

The average length of an answer to this question would be around 2/3 of a column in a 2-column page}
}

%% Question 3:
\newcommand{\questionThree} {
\youranswer{Question 3 - Describe Residual Connections (RC) by using equations. Consider both training and test time and explain how RC address the vanishing gradient problem.

The average length of an answer to this question would be around 1/2 of a column in a 2-column page}
}

%% Question 4:
\newcommand{\questionFour} {
\youranswer{Question 4 - Present and discuss the experiment results (all of the results and not just the ones you had to fill in) in Table 1 and Figures 4 and 5 (you may use any of the other Figures if you think they are relevant to your analysis). You will have to determine what data are relevant to the discussion, and what information can be extracted from it. Also, discuss what further experiments you would have ran on any combination of VGG08, VGG38, BN, RC in order to
\begin{itemize}
    \item Improve performance of the model trained (explain why you expect your suggested experiments will help with this).
    \item Learn more about the behaviour of BN and RC (explain what you are trying to learn and how).
\end{itemize}

The average length for an answer to this question is approximately 1 of the columns in a 2-column page}
}

%% Question 5:
\newcommand{\questionFive} {
\youranswer{Question 5 - Briefly draw your conclusions based on the results from the previous sections (what are the take-away messages?) and conclude your report with a recommendation for future work. 

Good recommendations for future work also draw on the broader literature (the papers already referenced are good starting points). Great recommendations for future work are not just incremental (an example of an incremental suggestion would be: "we could also train with different learning rates") but instead also identify meaningful questions or, in other words, questions with answers that might be somewhat more generally applicable. 

For example, \citep{huang2017densely} end with \begin{quote}``Because of their compact internal representations and reduced feature redundancy, DenseNets may be good feature extractors for various computer vision tasks that build on convolutional features, e.g.,  [4,5].''\end{quote} 

while \cite{bengio1993problem} state in their conclusions that \begin{quote}``There remains theoretical questions to be considered,  such as whether the problem with simple gradient descent  discussed in this paper would be observed with  chaotic attractors that are not  hyperbolic.\\\end{quote}

The length of this question description is indicative of the average length of a conclusion section}
}

%% - - - - - - - - - - - - FIGURES - - - - - - - - - - - - 

%% Question Figure 3:
\newcommand{\questionFigureThree} {
\youranswer{Question Figure 3 - Replace this image with a figure depicting the average gradient across layers, for the VGG38 model.
%
\begin{figure}[t]
    \centering
    \includegraphics[width=\linewidth]{example-image-duck}
    \caption{Gradient Flow on VGG38}
    \label{fig:grad_flow_38}
\end{figure}
}
}

%% Question Figure 4:
\newcommand{\questionFigureFour} {
\youranswer{Question Figure 4 - Replace this image with a figure depicting the training curves for the model with the best performance across experiments you have available. (Also edit the caption accordingly).
%
\begin{figure}[t]
    \centering
    \includegraphics[width=\linewidth]{example-image-duck}
    \caption{Training curves for ? ? ?}
    \label{fig:grad_flow_bestModel}
\end{figure}
}
}

%% Question Figure 5:
\newcommand{\questionFigureFive} {
\youranswer{Question Figure 5 - Replace this image with a figure depicting the average gradient across layers, for the model with the best performance across experiments you have available. (Also edit the caption accordingly).
%
\begin{figure}[t]
    \centering
    \includegraphics[width=\linewidth]{example-image-duck}
    \caption{Gradient Flow on ? ? ?}
    \label{fig:grad_flow_bestModel}
\end{figure}
}
}

%% - - - - - - - - - - - - TABLES - - - - - - - - - - - - 

%% Question Table 1:
\newcommand{\questionTableOne} {
\youranswer{
Question Table 1 - Fill in Table 1 with the results from your experiments on 
\begin{enumerate}
    \item \textit{VGG38 BN (LR 1e-3)}, and 
    \item \textit{VGG38 BN + RC (LR 1e-2)}.
\end{enumerate}
%
\begin{table*}[t]
    \centering
    \begin{tabular}{lr|ccccc}
    \toprule
        Model                   & LR   & \# Params & Train loss & Train acc & Val loss & Val acc \\
    \midrule
        VGG08                   & 1e-3 & 60 K      &  1.74      & 51.59     & 1.95     & 46.84 \\
        VGG38                   & 1e-3 & 336 K     &  4.61      & 00.01     & 4.61     & 00.01 \\
        VGG38 BN                & 1e-3 &     ?     &     ?      &     ?     &    ?     &     ? \\
        VGG38 RC                & 1e-3 & 336 K     &  1.33      & 61.52     & 1.84     & 52.32 \\
        VGG38 BN + RC           & 1e-3 & 339 K     &  1.26      & 62.99     & 1.73     & 53.76 \\
        VGG38 BN                & 1e-2 & 339 K     &  1.70      & 52.28     & 1.99     & 46.72 \\
        VGG38 BN + RC           & 1e-2 &     ?     &     ?      &     ?     &    ?     &     ? \\
    \bottomrule
    \end{tabular}
    \caption{Experiment results (number of model parameters, Training and Validation loss and accuracy) for different combinations of VGG08, VGG38, Batch Normalisation (BN), and Residual Connections (RC), LR is learning rate.}
    \label{tab:CIFAR_results}
\end{table*} 
}
}

%% END of YOUR ANSWERS

%% END of YOUR ANSWERS

%% Do not change anything in this file. Add your answers to mlp-cw1-questions.tex






\begin{document} 

\twocolumn[
\mlptitle{MLP Coursework 2}
\centerline{\studentNumber}
\vskip 7mm
]

\begin{abstract} 
Deep neural networks have become the state-of-the-art 
in many standard computer vision problems thanks to more powerful
neural networks and large labeled datasets.
While very deep networks allow for better deciphering
of the complex patterns in the data,
training these models successfully is a challenging task
due to problematic gradient flow through the layers, 
known as vanishing/exploding gradient problem (VGP and EGP respectively).
In this report, we first analyze this problem in VGG models
with 8 and 38 hidden layers on the CIFAR100 image dataset, 
by monitoring the gradient flow during training. 
We explore known solutions to this problem including batch
normalization or residual connections, and explain their theory
and implementation details. 
Our experiments show that batch normalization and residual connections effectively
address the aforementioned problem and hence enable a deeper model to outperform
shallower ones in the same experimental setup.
\end{abstract} 

\section{Introduction}
\label{sec:intro}
Despite the remarkable progress of deep neural networks in image classification problems~\cite{simonyan2014very, he2016deep}, training very deep networks is a challenging procedure.
One of the major problems is the VGP, a phenomenon where gradients from the loss function shrink to zero as they backpropagate
to earlier layers, hence preventing the network from updating its
weights effectively. 
This phenomenon is prevalent and has been extensively
studied in various deep network including feedforward  networks~\cite{glorot2010understanding}, 
RNNs~\cite{bengio1993problem}, and CNNs~\cite{he2016deep}. 
Multiple solutions have been proposed to mitigate this problem by using
weight initialization strategies~\cite{glorot2010understanding},
activation functions~\cite{glorot2010understanding},
input normalization~\cite{bishop1995neural},
batch normalization~\cite{ioffe2015batch}, and shortcut
connections \cite{he2016deep, huang2017densely}.

This report focuses on diagnosing the VGP occurred in the VGG38 model and addressing
it by implementing two standard solutions.
In particular, we first study the ``broken''
network in terms of its gradient flow, norm of gradients with respect to
model weights for each layer and contrast it 
to ones in the healthy VGG08 to pinpoint the problem.
Next, we review two standard solutions for this problem, 
batch normalization (BN)~\cite{ioffe2015batch} and residual connections (RC)~\cite{he2016deep}
in detail and discuss how they can address the gradient problem.
We first incorporate batch normalization (denoted as VGG38+BN), 
residual connections (denoted as VGG38+RC), 
and their combination (denoted as VGG38+BN+RC) to the given VGG38 architecture.
We train the resulting three configurations, and VGG08 and VGG38 models on 
CIFAR-100 dataset and present the results.
The results show that though separate use of BN and RC does tackle 
the vanishing/exploding gradient problem, therefore enabling the training of the VGG38 model, 
the best results are obtained by combining both BN and RC.

%


\section{Identifying training problems of a deep CNN}
\label{sec:task1}

\begin{figure}[t]
    \begin{subfigure}{\linewidth}
        \centering
        \includegraphics[width=\linewidth]{figures/loss_plot.pdf}
        \caption{Loss per epoch}
        \label{fig:loss_curves}
    \end{subfigure}

    \begin{subfigure}{\linewidth}
        \centering
        \includegraphics[width=\linewidth]{figures/accuracy_plot.pdf}
        \caption{Accuracy per epoch}
        \label{fig:acc_curves}
    \end{subfigure}
    \caption{Training curves for VGG08 and VGG38}
    \label{fig:curves}
\end{figure}

\begin{figure}[t]
    \centering
    \includegraphics[width=\linewidth]{figures/grad_flow_vgg08.pdf}
    \caption{Gradient flow on VGG08}
    \label{fig:grad_flow_08}
\end{figure}

\questionFigureThree

Concretely, training deep neural typically involves three steps, forward
pass, backward pass (or backpropagation algorithm~\cite{rumelhart1986learning}) and weight update.
The first step involves passing the input $x^0$ to the network and producing 
the network prediction and also the error value.
In detail, each layer takes in the output of the previous layer and applies
a non-linear transformation:
\begin{equation}
\label{eq.fprop}
\bx^{(l)} = f^{(l)}(\bx^{(l-1)}; W^{(l)})    
\end{equation} 
where $(l)$ denotes the $l$-th layer in $L$ layer deep network,
$f^{(l)}(\cdot,W^{(l)})$ is 
a non-linear transformation for layer $l$, and $W^{(l)})$ are the 
weights of layer $l$.
For instance, $f^{(l)}$ is typically a convolution operation followed by an activation
function in convolutional neural networks.
The second step involves the backpropagation algorithm, where we calculate
the gradient of an error function $E$ (e.g. cross-entropy) for each layer's
weight as follows:

\begin{equation}
    \label{eq.bprop}
\frac{\partial E}{\partial W^{(l)}} = \frac{\partial E}{\partial \bx^{(L)}} \frac{\partial \bx^{(L)}}{\partial \bx^{(L-1)}} \dots \frac{\partial \bx^{(l+1)}}{\partial \bx^{(l)}}\frac{\partial \bx^{(l)}}{\partial W^{(l)}}.
\end{equation}

This step includes consecutive tensor multiplications between multiple
partial derivative terms.
The final step involves updating model weights by using the computed 
$\frac{\partial E}{\partial W^{(l)}}$ with an update rule.
The exact update rule depends on the optimizer.

A notorious problem for training deep neural networks is the vanishing/exploding gradient
problem~\cite{bengio1993problem} that typically occurs in the backpropagation step when some of partial gradient terms in Eq.~\ref{eq.bprop} includes values larger or smaller than 1.
In this case, due to the multiple consecutive multiplications, the gradients w.r.t. weights
can get exponentially very small (close to 0) or very large (close to infinity) and
prevent effective learning of network weights.


%


Figures~\ref{fig:grad_flow_08} and \ref{fig:grad_flow_38} depict the gradient flows through
VGG architectures \cite{simonyan2014very} with 8 and 38 layers respectively,
trained and evaluated for a total of 100 epochs on the 
CIFAR100 dataset. \questionOne.


\section{Background Literature}
\label{sec:lit_rev}
In this section we will highlight some of the most influential
papers that have been central to overcoming the VGP in
deep CNNs.

\paragraph{Batch Normalization}\cite{ioffe2015batch}
BN seeks to solve the  problem of 
internal covariate shift (ICS), when distribution of each layer’s 
inputs changes during training, as the parameters of the previous layers change. 
The authors
argue that without batch normalization, the distribution of
each layer’s inputs can vary significantly due to the 
stochastic nature of randomly sampling mini-batches from your
training set. Layers in the network hence must continuously
adapt to these high variance distributions which hinders the
rate of convergence gradient-based optimizers. This optimization
problem is exacerbated further with network depth due
to the updating of parameters at layer $l$ being dependent on
the previous $l-1$ layers.

It is hence beneficial to embed the normalization of
training data into the network architecture after work from
LeCun \emph{et al.} showed that training converges faster with
this addition \cite{lecun2012efficient}. Through standardizing
the inputs to each layer, we take a step towards achieving
the fixed distributions of inputs that remove the ill effects
of ICS. Ioffe and Szegedy demonstrate the effectiveness of
their technique through training an ensemble of BN
networks which achieve an accuracy on the ImageNet classification
task exceeding that of humans in 14 times fewer
training steps than the state-of-the-art of the time.
It should be noted, however, that the exact reason for
BN’s effectiveness is still not completely understood and it is 
an open research question~\cite{santurkar2018does}.



\paragraph{Residual networks (ResNet)}\cite{he2016deep}
One interpretation of how the VGP arises is that stacking non-linear layers
between the input and output of networks makes the
connection between these variables increasingly
complex. This results in the gradients becoming
increasingly scrambled as they are propagated back through
the network and the desired mapping between input and output
being lost. He~\emph{et al.} observed this on 
a deep 56-layer neural network counter-intuitively
achieving a higher training error than a shallower 20-
layer network despite higher theoretical power.
Residual networks, colloquially
known as ResNets, aim to alleviate this through the
incorporation of skip connections that bypass the linear
transformations into the network architecture. The authors
argue that this new mapping is significantly easier
to optimize since if an identity mapping were optimal, the
network could comfortably learn to push the residual to
zero rather than attempting to fit an identity mapping via
a stack of nonlinear layers. They bolster their argument
by successfully training ResNets with depths exceeding
1000 layers on the CIFAR10 dataset.
Prior to their work, training even a 100-layer was accepted
as a great challenge within the deep learning community.
The addition of skip connections solves the VGP through
enabling information to flow more freely throughout the
network architecture without the addition of neither extra
parameters, nor computational complexity.

\section{Solution overview}
\subsection{Batch normalization}

\questionTwo.


\subsection{Residual connections}

\questionThree.


\section{Experiment Setup}

\questionFigureFour

\questionFigureFive

\questionTableOne

We conduct our experiment on the CIFAR-100 dataset \cite{krizhevsky2009learning},
which consists of 60,000 32x32 colour images from
100 different classes. The number of samples per class is balanced, and the
samples are split into training, validation, and test set while
maintaining balanced class proportions. In total, there are
47,500; 2,500; and 10,000 instances in the training, validation,
and test set, respectively. Moreover, we apply data
augmentation strategies (cropping, horizontal flipping) to
improve the generalization of the model.

With the goal of understanding whether BN or skip connections
help fighting vanishing gradients, we first test these
methods independently, before combining them in an attempt
to fully exploit the depth of the VGG38 model.

All experiments are conducted using the Adam optimizer with the default
learning rate (1e-3) -- unless otherwise specified, cosine annealing and a batch size of 100
for 100 epochs. 
Additionally, training images are augmented with random 
cropping and horizontal flipping.
Note that we do not use data augmentation at test time.
These hyperparameters along with the augmentation strategy are used
to produce the results shown in Figure~\ref{fig:curves}.

When used, BN is applied
after each convolutional layer, before the Leaky
ReLU non-linearity. Similarly, the skip connections are applied from 
before the convolution layer to before the final activation function
of the block as per Figure~2 of \cite{he2016deep} 





\section{Results and Discussion}
\label{sec:disc}

\questionFour.

\section{Conclusion}
\label{sec:concl}
    
\questionFive.

\bibliography{refs}

\end{document} 





